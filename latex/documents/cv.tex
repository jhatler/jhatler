%!TEX TS-program = xelatex
%!TEX encoding = UTF-8 Unicode
% Awesome CV LaTeX Template for CV/Resume
%
% This template has been downloaded from:
% https://github.com/posquit0/Awesome-CV
%
% Author:
% Claud D. Park <posquit0.bj@gmail.com>
% http://www.posquit0.com
%
% Template license:
% CC BY-SA 4.0 (https://creativecommons.org/licenses/by-sa/4.0/)
%


%-------------------------------------------------------------------------------
% CONFIGURATIONS
%-------------------------------------------------------------------------------
% A4 paper size by default, use 'letterpaper' for US letter
\documentclass[11pt, letterpaper]{awesome-cv}

% Configure page margins with geometry
\geometry{left=1.4cm, top=.8cm, right=1.4cm, bottom=1.8cm, footskip=.5cm}

% Color for highlights
\colorlet{awesome}{awesome-red}

% Set false if you don't want to highlight section with awesome color
\setbool{acvSectionColorHighlight}{true}

% If you would like to change the social information separator from a pipe (|) to something else
\renewcommand{\acvHeaderSocialSep}{\quad\textbar\quad}

% Tune hyphenation
\righthyphenmin=5
\lefthyphenmin=5

%-------------------------------------------------------------------------------
%	PERSONAL INFORMATION
%	Comment any of the lines below if they are not required
%-------------------------------------------------------------------------------
\name{Jaremy A.}{Hatler}
\position{DevSecOps Engineer{\enskip\cdotp\enskip}Platform Engineer{\enskip\cdotp\enskip}Site Reliability Engineer{\enskip\cdotp\enskip}Solutions Architect{\enskip\cdotp\enskip}US Citizen{\enskip\cdotp\enskip}Eligible for Security Clearance}
\address{967 Idaho Ave, Akron, Ohio, 44314, United States}

\mobile{(+1) 234-255-2438}
\email{root@jhatler.com}
\github{jhatler}
\cv{cv.jhatler.com}

\quote{``Simple can be harder than complex. You have to work hard to get your thinking clean.'' --- Steve Jobs}


%-------------------------------------------------------------------------------
\begin{document}

% Print the header with above personal information
% Give optional argument to change alignment(C: center, L: left, R: right)
\makecvheader[C]

% Print the footer with 3 arguments(<left>, <center>, <right>)
% Leave any of these blank if they are not needed
\makecvfooter
  {\today}
  {~~~·~~~Jaremy A. Hatler~~~·~~~Eligible for US Security Clearance}
  {\thepage}


%-------------------------------------------------------------------------------
%	CV/RESUME CONTENT
%	Each section is imported separately, open each file in turn to modify content
%-------------------------------------------------------------------------------
\cvsection{Experience}
\begin{cventries}
    \cventry
        { Founder \& Consultant }
        { Enacted Services }
        { Akron, OH }
        { Jul. 2024 --- Present }
        {
            \begin{cvitems}
                \item{\textbf{Bootstrapped a hardened, self-hosting Linux distro} (Gentoo/systemd-based) anchored in cryptographic boot measurements, transparent multi-factor encryption, and immutability, \textbf{balancing the mitigation of Advanced Persistent Threats (APTs) with developer ease-of-use}.}
                \item{\textbf{Developed scripts and documentation to deploy YubiKeys across developer workflows} (FIDO2, PIV, OpenPGP), securing day-to-day cryptographic operations (STRIDE threat model), \textbf{improving developer identity assurance and reducing toil}.}
                \item{\textbf{Evaluated agentic AI, vector database, GPU virtualization (MLOps), and BaaS solutions} (e.g., LangChain, Weaviate, Firebase) for a client's early product ideation, orchestrating PoCs that \textbf{ clarified technology choices within a broader architectural strategy} focused on security, reliability, performance, and future scalability.}
                \item{\textbf{Engineered a turnkey internal development platform (IDP)} using AWS Amplify, CloudWatch/X-Ray, Terraform, and GitHub Actions (GitOps), \textbf{bootstrapping observability} and instituting core security principles (least privilege, secure-by-default, etc.) in the infrastructure with a \textbf{shift-left AppSec approach} (SAST, DAST, CSPM), \textbf{effectively limiting technical debt and risk from project inception}.}
            \end{cvitems}
        }
\end{cventries}
\begin{cventries}
    \cventry
        { DevOps Engineer III (Engineering --- Lead) }
        { Ezurio, Inc. (Formerly Laird Connectivity, Inc.) }
        { Akron, OH }
        { Jan. 2023 --- Jul. 2024 }
        {
            \begin{cvitems}
                \item{\textbf{Oversaw and collaborated with a transformation partner} to assess DevOps maturity and \textbf{map value streams} across the engineering organization, delivering a roadmap that launched a company-wide \textbf{DevOps Modernization initiative and Agile transformation}.}
                \item{\textbf{Architected and implemented an Internal Development Platform (IDP)} with Spacelift, Ansible, Terraform, Docker, Devcontainers, Packer, and GitHub Actions, incorporating \textbf{shift-left security} features (SCA, CSPM, SAST, DAST, policy-as-code), \textbf{GitOps} release management, and real-time cost monitoring while ensuring \textbf{SLSA level 3} artifact provenance for end-to-end software supply chain integrity.\hfill{\href{https://github.com/jhatler/janus}{\textbf{{\color{awesome}Git}Hub Project:} jhatler/janus}}}
                \item{\textbf{Migrated all on-prem DevOps systems, data, and pipelines to AWS} through leveraging the IDP, retiring risky on-prem infrastructure, \textbf{saving \$21,000/year} in infrastructure costs, \textbf{reducing DevOps workload by 50\%}, and enhancing system resilience; \textbf{effectively reducing high-severity, reactive issue resolution} in favor of a more deliberate, error-budget-aware approach.}
                \item{\textbf{Facilitated agile coaching} across engineering and PMO, fostering an iterative Plan-Do-Check-Act (PDCA) approach, cultivating an Agile Community of Practice, and aligning enterprise teams around new tools and processes to \textbf{ensure scaling of DevOps principles and responsiveness to evolving team needs} throughout the transformation.}
                \item{\textbf{Onboarded and mentored a new DevOps Engineer} to gain deep knowledge of our diverse ecosystem and become the primary DevOps SME for the organization, \textbf{ensuring continuity of expertise within 18 months}.}
            \end{cvitems}
        }
\end{cventries}
\begin{cventries}
    \cventry
        { DevOps Engineer III (IT) }
        { Laird Connectivity, Inc. }
        { Akron, OH }
        { Jul. 2019 --- Jan. 2023 }
        {
            \begin{cvitems}
                \item{\textbf{Engineered a custom Jenkins (Groovy) framework for embedded Linux BSPs} (Buildroot/Yocto) to declaratively manage build parameters and dependencies \textbf{across 1000+ Git repositories}, enabling advanced versioning and branching strategies, unifying developer experience, \textbf{reducing toil, and increasing deployment frequency} from the order of days to hours.}
                \item{\textbf{Orchestrated a comprehensive domain migration} for all DevOps systems, developing a suite of custom tools for data analysis, staging/UAT setup, migration, validation, and rollback to ensure every reference (documentation, 1500+ Git repos, code reviews, issues) was transparently updated and referential integrity was maintained \textbf{without impacting any release cadence during corporate spin-off} (exceeding SLOs).}
                \item{\textbf{Established a 200-core, 1.5TB RAM bare-metal cluster} with Rancher (Kubernetes), Harvester, Datadog, Docker, and iPXE to which all high-impact CI/CD workloads were migrated, \textbf{minimizing performance contention and decreasing lead time for changes} through increased parallelism to align with multi-year strategic goals.}
                \item{\textbf{Organized ``Lunch \& Learn'' sessions} averaging approx. 30 participants per event, intermixing guest speakers with a series of Git Clinics I developed covering deep dives on Git features (rebasing, cherry-picking, etc.) and \textbf{reinforcing DevOps best practices}.}
                \item{\textbf{Organized and directed a 12-hour DevOps Modernization Workshop} that brought together product managers, executives, and technical leads to align on topics synthesized from the DevOps Research and Assessment \textbf{(DORA)} framework, The Open Group Architecture Framework \textbf{(TOGAF)}, The Open Group Service Integration Maturity Model \textbf{(OSIMM)}, the Digital Practitioner Body of Knowledge \textbf{(DPBoK)}, and DevSecOps research from Carnegie Mellon University Software Engineering Institute \textbf{(CMU-SEI)}, successfully \textbf{culminating in executive buy-in} to onboard a specialized transformation partner \textbf{for ambitious development platform upgrades to accelerate time-to-market}.}
            \end{cvitems}
        }
\end{cventries}
\begin{cventries}
    \cventry
        { DevOps Engineer II (IT) }
        { Laird Connectivity, Inc. (Laird PLC Spin-off) }
        { Akron, OH }
        { Apr. 2018 --- Jul. 2019 }
        {
            \begin{cvitems}
                \item{\textbf{Oversaw an in-house migration} of digital marketing systems to AWS, performing a comprehensive architectural/security review of the existing vendor solution, \textbf{orchestrating the migration without downtime}, successfully integrating Dynatrace for observability, \textbf{improving site reliability under high-traffic conditions, and accelerating time-to-market for new campaigns}.}
                \item{\textbf{Developed custom tools to convert 300+ PlasticSCM repositories to Git}, preserving full history/metadata, \textbf{saving \$5k/year} on licensing, and enabling the professional services team to integrate into the central DevOps platform.}
                \item{\textbf{Authored and delivered an on-site Git training curriculum for 30+ engineers}, enhancing collaboration with the wider engineering organization and \textbf{motivating usage of shared automation}.}
                \item{\textbf{Owned the DevOps platform during two business-unit divestitures}, leveraging Infrastructure as Code to create parallel systems \textbf{without additional capex} and implementing Jira Service Desk to support ITSM within the organization, \textbf{exceeding SLOs with zero support tickets} under Transition Service Agreements.}
            \end{cvitems}
        }
\end{cventries}
\begin{cventries}
    \cventry
        { DevOps Engineer I (IT) }
        { Laird PLC }
        { Akron, OH }
        { May. 2016 --- Apr. 2018 }
        {
            \begin{cvitems}
                \item{\textbf{Transformed an undocumented legacy DevOps ecosystem} into a centralized platform (Jenkins, GitLab, Bugzilla, JFrog Artifactory, MediaWiki) by documenting workflows, creating job aids, tightening integrations, and executing a \textbf{zero-downtime P2V migration}.}
                \item{\textbf{Expanded DevOps systems to support 100+ newly acquired staff}, migrating data from Jira, SVN, and Redmine into Bugzilla, GitLab, and other platform tools and \textbf{coaching teams to adopt standardized processes}.}
                \item{\textbf{Engineered containerized DevOps environments} (Docker, Packer, Terraform, Rancher) to \textbf{support M\&A activities and ensure business continuity}, achieving rapid recovery during a hyperconverged infrastructure failure with no data loss or missed deadlines.}
                \item{\textbf{Engineered cached Android BSP pipelines} (Docker, BTRFS, Jenkins) following extensive analysis of the Android build system and load testing of available infrastructure, successfully \textbf{cutting build times by 80\% and reducing storage consumption by 40\%} while preserving artifact integrity.}
            \end{cvitems}
        }
\end{cventries}
\begin{cventries}
    \cventry
        { AWS Systems Administrator }
        { My1HR, Inc. }
        { Akron, OH }
        { Mar. 2013 --- Feb. 2016 }
        {
            \begin{cvitems}
                \item{\textbf{Architected and scaled a secure, multi-tier AWS infrastructure} (EC2, RDS, ELB, S3, CloudFront) with \textbf{layered security controls} (IDP, DLP, SIEM) and \textbf{tightly integrated observability tools} (ELK, MySQL NDB Cluster Manager, NewRelic, Zabbix, Nagios) that powered a high-traffic, multi-tenant, HIPAA/HITECH/PCI-DSS compliant SaaS product which helped over \textbf{1M+ families annually} to enroll in healthcare plans --- \textbf{exceeding SLAs by maintaining five-nines availability during peak seasonal traffic}.}
                \item{\textbf{Diagnosed and resolved critical MySQL NDB Cluster CGE crashes} (SEGFAULTS) by analyzing core dumps to pinpoint a memory page alignment bug and working with Oracle to deploy a patch within one week, \textbf{maintaining all SLOs and SLAs during the process}.}
                \item{\textbf{Streamlined infrastructure management} by migrating from Chef/RightScale to CloudFormation/Puppet, improving autoscaling agility by 40\% and \textbf{saving \$50k/year}.}
                \item{\textbf{Evangelized Docker for local development}, standardizing tooling across a 30-person team and \textbf{saving 20+ hours} during developer onboarding.}
            \end{cvitems}
        }
\end{cventries}


\cvsection{Experience}
\begin{cventries}
    \cventry
        { Founder \& Consultant }
        { Enacted Services }
        { Akron, OH }
        { Jul. 2024 --- Present }
        {
            \begin{cvitems}
                \item{\textbf{Bootstrapped a hardened, self-hosting Linux distro} (Gentoo/systemd-based) anchored in cryptographic boot measurements, transparent multi-factor encryption, and immutability, \textbf{balancing the mitigation of Advanced Persistent Threats (APTs) with developer ease-of-use}.}
                \item{\textbf{Developed scripts and documentation to deploy YubiKeys across developer workflows} (FIDO2, PIV, OpenPGP), securing day-to-day cryptographic operations (STRIDE threat model), \textbf{improving developer identity assurance and reducing toil}.}
                \item{\textbf{Evaluated agentic AI, vector database, GPU virtualization (MLOps), and BaaS solutions} (e.g., LangChain, Weaviate, Firebase) for a client's early product ideation, orchestrating PoCs that \textbf{ clarified technology choices within a broader architectural strategy} focused on security, reliability, performance, and future scalability.}
                \item{\textbf{Engineered a turnkey internal development platform (IDP)} using AWS Amplify, CloudWatch/X-Ray, Terraform, and GitHub Actions (GitOps), \textbf{bootstrapping observability} and instituting core security principles (least privilege, secure-by-default, etc.) in the infrastructure with a \textbf{shift-left AppSec approach} (SAST, DAST, CSPM), \textbf{effectively limiting technical debt and risk from project inception}.}
            \end{cvitems}
        }
\end{cventries}
\begin{cventries}
    \cventry
        { DevOps Engineer III (Engineering --- Lead) }
        { Ezurio, Inc. (Formerly Laird Connectivity, Inc.) }
        { Akron, OH }
        { Jan. 2023 --- Jul. 2024 }
        {
            \begin{cvitems}
                \item{\textbf{Oversaw and collaborated with a transformation partner} to assess DevOps maturity and \textbf{map value streams} across the engineering organization, delivering a roadmap that launched a company-wide \textbf{DevOps Modernization initiative and Agile transformation}.}
                \item{\textbf{Architected and implemented an Internal Development Platform (IDP)} with Spacelift, Ansible, Terraform, Docker, Devcontainers, Packer, and GitHub Actions, incorporating \textbf{shift-left security} features (SCA, CSPM, SAST, DAST, policy-as-code), \textbf{GitOps} release management, and real-time cost monitoring while ensuring \textbf{SLSA level 3} artifact provenance for end-to-end software supply chain integrity.\hfill{\href{https://github.com/jhatler/janus}{\textbf{{\color{awesome}Git}Hub Project:} jhatler/janus}}}
                \item{\textbf{Migrated all on-prem DevOps systems, data, and pipelines to AWS} through leveraging the IDP, retiring risky on-prem infrastructure, \textbf{saving \$21,000/year} in infrastructure costs, \textbf{reducing DevOps workload by 50\%}, and enhancing system resilience; \textbf{effectively reducing high-severity, reactive issue resolution} in favor of a more deliberate, error-budget-aware approach.}
                \item{\textbf{Facilitated agile coaching} across engineering and PMO, fostering an iterative Plan-Do-Check-Act (PDCA) approach, cultivating an Agile Community of Practice, and aligning enterprise teams around new tools and processes to \textbf{ensure scaling of DevOps principles and responsiveness to evolving team needs} throughout the transformation.}
                \item{\textbf{Onboarded and mentored a new DevOps Engineer} to gain deep knowledge of our diverse ecosystem and become the primary DevOps SME for the organization, \textbf{ensuring continuity of expertise within 18 months}.}
            \end{cvitems}
        }
\end{cventries}
\begin{cventries}
    \cventry
        { DevOps Engineer III (IT) }
        { Laird Connectivity, Inc. }
        { Akron, OH }
        { Jul. 2019 --- Jan. 2023 }
        {
            \begin{cvitems}
                \item{\textbf{Engineered a custom Jenkins (Groovy) framework for embedded Linux BSPs} (Buildroot/Yocto) to declaratively manage build parameters and dependencies \textbf{across 1000+ Git repositories}, enabling advanced versioning and branching strategies, unifying developer experience, \textbf{reducing toil, and increasing deployment frequency} from the order of days to hours.}
                \item{\textbf{Orchestrated a comprehensive domain migration} for all DevOps systems, developing a suite of custom tools for data analysis, staging/UAT setup, migration, validation, and rollback to ensure every reference (documentation, 1500+ Git repos, code reviews, issues) was transparently updated and referential integrity was maintained \textbf{without impacting any release cadence during corporate spin-off} (exceeding SLOs).}
                \item{\textbf{Established a 200-core, 1.5TB RAM bare-metal cluster} with Rancher (Kubernetes), Harvester, Datadog, Docker, and iPXE to which all high-impact CI/CD workloads were migrated, \textbf{minimizing performance contention and decreasing lead time for changes} through increased parallelism to align with multi-year strategic goals.}
                \item{\textbf{Organized ``Lunch \& Learn'' sessions} averaging approx. 30 participants per event, intermixing guest speakers with a series of Git Clinics I developed covering deep dives on Git features (rebasing, cherry-picking, etc.) and \textbf{reinforcing DevOps best practices}.}
                \item{\textbf{Organized and directed a 12-hour DevOps Modernization Workshop} that brought together product managers, executives, and technical leads to align on topics synthesized from the DevOps Research and Assessment \textbf{(DORA)} framework, The Open Group Architecture Framework \textbf{(TOGAF)}, The Open Group Service Integration Maturity Model \textbf{(OSIMM)}, the Digital Practitioner Body of Knowledge \textbf{(DPBoK)}, and DevSecOps research from Carnegie Mellon University Software Engineering Institute \textbf{(CMU-SEI)}, successfully \textbf{culminating in executive buy-in} to onboard a specialized transformation partner \textbf{for ambitious development platform upgrades to accelerate time-to-market}.}
            \end{cvitems}
        }
\end{cventries}
\begin{cventries}
    \cventry
        { DevOps Engineer II (IT) }
        { Laird Connectivity, Inc. (Laird PLC Spin-off) }
        { Akron, OH }
        { Apr. 2018 --- Jul. 2019 }
        {
            \begin{cvitems}
                \item{\textbf{Oversaw an in-house migration} of digital marketing systems to AWS, performing a comprehensive architectural/security review of the existing vendor solution, \textbf{orchestrating the migration without downtime}, successfully integrating Dynatrace for observability, \textbf{improving site reliability under high-traffic conditions, and accelerating time-to-market for new campaigns}.}
                \item{\textbf{Developed custom tools to convert 300+ PlasticSCM repositories to Git}, preserving full history/metadata, \textbf{saving \$5k/year} on licensing, and enabling the professional services team to integrate into the central DevOps platform.}
                \item{\textbf{Authored and delivered an on-site Git training curriculum for 30+ engineers}, enhancing collaboration with the wider engineering organization and \textbf{motivating usage of shared automation}.}
                \item{\textbf{Owned the DevOps platform during two business-unit divestitures}, leveraging Infrastructure as Code to create parallel systems \textbf{without additional capex} and implementing Jira Service Desk to support ITSM within the organization, \textbf{exceeding SLOs with zero support tickets} under Transition Service Agreements.}
            \end{cvitems}
        }
\end{cventries}
\begin{cventries}
    \cventry
        { DevOps Engineer I (IT) }
        { Laird PLC }
        { Akron, OH }
        { May. 2016 --- Apr. 2018 }
        {
            \begin{cvitems}
                \item{\textbf{Transformed an undocumented legacy DevOps ecosystem} into a centralized platform (Jenkins, GitLab, Bugzilla, JFrog Artifactory, MediaWiki) by documenting workflows, creating job aids, tightening integrations, and executing a \textbf{zero-downtime P2V migration}.}
                \item{\textbf{Expanded DevOps systems to support 100+ newly acquired staff}, migrating data from Jira, SVN, and Redmine into Bugzilla, GitLab, and other platform tools and \textbf{coaching teams to adopt standardized processes}.}
                \item{\textbf{Engineered containerized DevOps environments} (Docker, Packer, Terraform, Rancher) to \textbf{support M\&A activities and ensure business continuity}, achieving rapid recovery during a hyperconverged infrastructure failure with no data loss or missed deadlines.}
                \item{\textbf{Engineered cached Android BSP pipelines} (Docker, BTRFS, Jenkins) following extensive analysis of the Android build system and load testing of available infrastructure, successfully \textbf{cutting build times by 80\% and reducing storage consumption by 40\%} while preserving artifact integrity.}
            \end{cvitems}
        }
\end{cventries}
\begin{cventries}
    \cventry
        { AWS Systems Administrator }
        { My1HR, Inc. }
        { Akron, OH }
        { Mar. 2013 --- Feb. 2016 }
        {
            \begin{cvitems}
                \item{\textbf{Architected and scaled a secure, multi-tier AWS infrastructure} (EC2, RDS, ELB, S3, CloudFront) with \textbf{layered security controls} (IDP, DLP, SIEM) and \textbf{tightly integrated observability tools} (ELK, MySQL NDB Cluster Manager, NewRelic, Zabbix, Nagios) that powered a high-traffic, multi-tenant, HIPAA/HITECH/PCI-DSS compliant SaaS product which helped over \textbf{1M+ families annually} to enroll in healthcare plans --- \textbf{exceeding SLAs by maintaining five-nines availability during peak seasonal traffic}.}
                \item{\textbf{Diagnosed and resolved critical MySQL NDB Cluster CGE crashes} (SEGFAULTS) by analyzing core dumps to pinpoint a memory page alignment bug and working with Oracle to deploy a patch within one week, \textbf{maintaining all SLOs and SLAs during the process}.}
                \item{\textbf{Streamlined infrastructure management} by migrating from Chef/RightScale to CloudFormation/Puppet, improving autoscaling agility by 40\% and \textbf{saving \$50k/year}.}
                \item{\textbf{Evangelized Docker for local development}, standardizing tooling across a 30-person team and \textbf{saving 20+ hours} during developer onboarding.}
            \end{cvitems}
        }
\end{cventries}


\cvsection{Experience}
\begin{cventries}
    \cventry
        { Founder \& Consultant }
        { Enacted Services }
        { Akron, OH }
        { Jul. 2024 --- Present }
        {
            \begin{cvitems}
                \item{\textbf{Bootstrapped a hardened, self-hosting Linux distro} (Gentoo/systemd-based) anchored in cryptographic boot measurements, transparent multi-factor encryption, and immutability, \textbf{balancing the mitigation of Advanced Persistent Threats (APTs) with developer ease-of-use}.}
                \item{\textbf{Developed scripts and documentation to deploy YubiKeys across developer workflows} (FIDO2, PIV, OpenPGP), securing day-to-day cryptographic operations (STRIDE threat model), \textbf{improving developer identity assurance and reducing toil}.}
                \item{\textbf{Evaluated agentic AI, vector database, GPU virtualization (MLOps), and BaaS solutions} (e.g., LangChain, Weaviate, Firebase) for a client's early product ideation, orchestrating PoCs that \textbf{ clarified technology choices within a broader architectural strategy} focused on security, reliability, performance, and future scalability.}
                \item{\textbf{Engineered a turnkey internal development platform (IDP)} using AWS Amplify, CloudWatch/X-Ray, Terraform, and GitHub Actions (GitOps), \textbf{bootstrapping observability} and instituting core security principles (least privilege, secure-by-default, etc.) in the infrastructure with a \textbf{shift-left AppSec approach} (SAST, DAST, CSPM), \textbf{effectively limiting technical debt and risk from project inception}.}
            \end{cvitems}
        }
\end{cventries}
\begin{cventries}
    \cventry
        { DevOps Engineer III (Engineering --- Lead) }
        { Ezurio, Inc. (Formerly Laird Connectivity, Inc.) }
        { Akron, OH }
        { Jan. 2023 --- Jul. 2024 }
        {
            \begin{cvitems}
                \item{\textbf{Oversaw and collaborated with a transformation partner} to assess DevOps maturity and \textbf{map value streams} across the engineering organization, delivering a roadmap that launched a company-wide \textbf{DevOps Modernization initiative and Agile transformation}.}
                \item{\textbf{Architected and implemented an Internal Development Platform (IDP)} with Spacelift, Ansible, Terraform, Docker, Devcontainers, Packer, and GitHub Actions, incorporating \textbf{shift-left security} features (SCA, CSPM, SAST, DAST, policy-as-code), \textbf{GitOps} release management, and real-time cost monitoring while ensuring \textbf{SLSA level 3} artifact provenance for end-to-end software supply chain integrity.\hfill{\href{https://github.com/jhatler/janus}{\textbf{{\color{awesome}Git}Hub Project:} jhatler/janus}}}
                \item{\textbf{Migrated all on-prem DevOps systems, data, and pipelines to AWS} through leveraging the IDP, retiring risky on-prem infrastructure, \textbf{saving \$21,000/year} in infrastructure costs, \textbf{reducing DevOps workload by 50\%}, and enhancing system resilience; \textbf{effectively reducing high-severity, reactive issue resolution} in favor of a more deliberate, error-budget-aware approach.}
                \item{\textbf{Facilitated agile coaching} across engineering and PMO, fostering an iterative Plan-Do-Check-Act (PDCA) approach, cultivating an Agile Community of Practice, and aligning enterprise teams around new tools and processes to \textbf{ensure scaling of DevOps principles and responsiveness to evolving team needs} throughout the transformation.}
                \item{\textbf{Onboarded and mentored a new DevOps Engineer} to gain deep knowledge of our diverse ecosystem and become the primary DevOps SME for the organization, \textbf{ensuring continuity of expertise within 18 months}.}
            \end{cvitems}
        }
\end{cventries}
\begin{cventries}
    \cventry
        { DevOps Engineer III (IT) }
        { Laird Connectivity, Inc. }
        { Akron, OH }
        { Jul. 2019 --- Jan. 2023 }
        {
            \begin{cvitems}
                \item{\textbf{Engineered a custom Jenkins (Groovy) framework for embedded Linux BSPs} (Buildroot/Yocto) to declaratively manage build parameters and dependencies \textbf{across 1000+ Git repositories}, enabling advanced versioning and branching strategies, unifying developer experience, \textbf{reducing toil, and increasing deployment frequency} from the order of days to hours.}
                \item{\textbf{Orchestrated a comprehensive domain migration} for all DevOps systems, developing a suite of custom tools for data analysis, staging/UAT setup, migration, validation, and rollback to ensure every reference (documentation, 1500+ Git repos, code reviews, issues) was transparently updated and referential integrity was maintained \textbf{without impacting any release cadence during corporate spin-off} (exceeding SLOs).}
                \item{\textbf{Established a 200-core, 1.5TB RAM bare-metal cluster} with Rancher (Kubernetes), Harvester, Datadog, Docker, and iPXE to which all high-impact CI/CD workloads were migrated, \textbf{minimizing performance contention and decreasing lead time for changes} through increased parallelism to align with multi-year strategic goals.}
                \item{\textbf{Organized ``Lunch \& Learn'' sessions} averaging approx. 30 participants per event, intermixing guest speakers with a series of Git Clinics I developed covering deep dives on Git features (rebasing, cherry-picking, etc.) and \textbf{reinforcing DevOps best practices}.}
                \item{\textbf{Organized and directed a 12-hour DevOps Modernization Workshop} that brought together product managers, executives, and technical leads to align on topics synthesized from the DevOps Research and Assessment \textbf{(DORA)} framework, The Open Group Architecture Framework \textbf{(TOGAF)}, The Open Group Service Integration Maturity Model \textbf{(OSIMM)}, the Digital Practitioner Body of Knowledge \textbf{(DPBoK)}, and DevSecOps research from Carnegie Mellon University Software Engineering Institute \textbf{(CMU-SEI)}, successfully \textbf{culminating in executive buy-in} to onboard a specialized transformation partner \textbf{for ambitious development platform upgrades to accelerate time-to-market}.}
            \end{cvitems}
        }
\end{cventries}
\begin{cventries}
    \cventry
        { DevOps Engineer II (IT) }
        { Laird Connectivity, Inc. (Laird PLC Spin-off) }
        { Akron, OH }
        { Apr. 2018 --- Jul. 2019 }
        {
            \begin{cvitems}
                \item{\textbf{Oversaw an in-house migration} of digital marketing systems to AWS, performing a comprehensive architectural/security review of the existing vendor solution, \textbf{orchestrating the migration without downtime}, successfully integrating Dynatrace for observability, \textbf{improving site reliability under high-traffic conditions, and accelerating time-to-market for new campaigns}.}
                \item{\textbf{Developed custom tools to convert 300+ PlasticSCM repositories to Git}, preserving full history/metadata, \textbf{saving \$5k/year} on licensing, and enabling the professional services team to integrate into the central DevOps platform.}
                \item{\textbf{Authored and delivered an on-site Git training curriculum for 30+ engineers}, enhancing collaboration with the wider engineering organization and \textbf{motivating usage of shared automation}.}
                \item{\textbf{Owned the DevOps platform during two business-unit divestitures}, leveraging Infrastructure as Code to create parallel systems \textbf{without additional capex} and implementing Jira Service Desk to support ITSM within the organization, \textbf{exceeding SLOs with zero support tickets} under Transition Service Agreements.}
            \end{cvitems}
        }
\end{cventries}
\begin{cventries}
    \cventry
        { DevOps Engineer I (IT) }
        { Laird PLC }
        { Akron, OH }
        { May. 2016 --- Apr. 2018 }
        {
            \begin{cvitems}
                \item{\textbf{Transformed an undocumented legacy DevOps ecosystem} into a centralized platform (Jenkins, GitLab, Bugzilla, JFrog Artifactory, MediaWiki) by documenting workflows, creating job aids, tightening integrations, and executing a \textbf{zero-downtime P2V migration}.}
                \item{\textbf{Expanded DevOps systems to support 100+ newly acquired staff}, migrating data from Jira, SVN, and Redmine into Bugzilla, GitLab, and other platform tools and \textbf{coaching teams to adopt standardized processes}.}
                \item{\textbf{Engineered containerized DevOps environments} (Docker, Packer, Terraform, Rancher) to \textbf{support M\&A activities and ensure business continuity}, achieving rapid recovery during a hyperconverged infrastructure failure with no data loss or missed deadlines.}
                \item{\textbf{Engineered cached Android BSP pipelines} (Docker, BTRFS, Jenkins) following extensive analysis of the Android build system and load testing of available infrastructure, successfully \textbf{cutting build times by 80\% and reducing storage consumption by 40\%} while preserving artifact integrity.}
            \end{cvitems}
        }
\end{cventries}
\begin{cventries}
    \cventry
        { AWS Systems Administrator }
        { My1HR, Inc. }
        { Akron, OH }
        { Mar. 2013 --- Feb. 2016 }
        {
            \begin{cvitems}
                \item{\textbf{Architected and scaled a secure, multi-tier AWS infrastructure} (EC2, RDS, ELB, S3, CloudFront) with \textbf{layered security controls} (IDP, DLP, SIEM) and \textbf{tightly integrated observability tools} (ELK, MySQL NDB Cluster Manager, NewRelic, Zabbix, Nagios) that powered a high-traffic, multi-tenant, HIPAA/HITECH/PCI-DSS compliant SaaS product which helped over \textbf{1M+ families annually} to enroll in healthcare plans --- \textbf{exceeding SLAs by maintaining five-nines availability during peak seasonal traffic}.}
                \item{\textbf{Diagnosed and resolved critical MySQL NDB Cluster CGE crashes} (SEGFAULTS) by analyzing core dumps to pinpoint a memory page alignment bug and working with Oracle to deploy a patch within one week, \textbf{maintaining all SLOs and SLAs during the process}.}
                \item{\textbf{Streamlined infrastructure management} by migrating from Chef/RightScale to CloudFormation/Puppet, improving autoscaling agility by 40\% and \textbf{saving \$50k/year}.}
                \item{\textbf{Evangelized Docker for local development}, standardizing tooling across a 30-person team and \textbf{saving 20+ hours} during developer onboarding.}
            \end{cvitems}
        }
\end{cventries}



\cvsection{Projects}
\begin{cventries}
    \cventry
        { Creator and Maintainer }
        { JANUS }
        { \href{https://github.com/jhatler/janus}{\textbf{GitHub}: jhatler/janus}
 }
        { May 2023 --- Present }
        {
          Internal Development Platform (IDP) targeting complex multi-cloud environments such as AI/ML workloads, IoT Device Management, OS distribution and support, etc. Utilizes Spacelift, Ansible, Terraform, Packer, Docker, Devcontainers, and GitHub Actions/Codespaces to provide a fully integrated, loosely coupled development environment. Supports Ubuntu 8.04.4 and up to aid in migration of legacy systems to modern platforms. Integrates Aikido, Codacy, and Infracost for key security, quality, and cost management features. Manages production workloads at multiple businesses.
        }
\end{cventries}
\begin{cventries}
    \cventry
        { Contributor }
        { Gentoo Linux }
        { \href{https://www.gentoo.org/}{\textbf{Homepage}: gentoo.org}
 }
        { 2018 --- Present }
        {
          Routine contributor to the Gentoo Linux project, a source-based Linux distribution with a focus on flexibility and customization. Normal contributions include assisting the community troubleshoot package build failures via IRC and email, triaging bugs found in my weekly builds of the Gentoo tree (covering approximately 8,000 packages on amd64, arm64), and testing patches for project maintainers.
        }
\end{cventries}

\cvsection{Experience}
\begin{cventries}
    \cventry
        { Founder \& Consultant }
        { Enacted Services }
        { Akron, OH }
        { Jul. 2024 --- Present }
        {
            \begin{cvitems}
                \item{\textbf{Bootstrapped a hardened, self-hosting Linux distro} (Gentoo/systemd-based) anchored in cryptographic boot measurements, transparent multi-factor encryption, and immutability, \textbf{balancing the mitigation of Advanced Persistent Threats (APTs) with developer ease-of-use}.}
                \item{\textbf{Developed scripts and documentation to deploy YubiKeys across developer workflows} (FIDO2, PIV, OpenPGP), securing day-to-day cryptographic operations (STRIDE threat model), \textbf{improving developer identity assurance and reducing toil}.}
                \item{\textbf{Evaluated agentic AI, vector database, GPU virtualization (MLOps), and BaaS solutions} (e.g., LangChain, Weaviate, Firebase) for a client's early product ideation, orchestrating PoCs that \textbf{ clarified technology choices within a broader architectural strategy} focused on security, reliability, performance, and future scalability.}
                \item{\textbf{Engineered a turnkey internal development platform (IDP)} using AWS Amplify, CloudWatch/X-Ray, Terraform, and GitHub Actions (GitOps), \textbf{bootstrapping observability} and instituting core security principles (least privilege, secure-by-default, etc.) in the infrastructure with a \textbf{shift-left AppSec approach} (SAST, DAST, CSPM), \textbf{effectively limiting technical debt and risk from project inception}.}
            \end{cvitems}
        }
\end{cventries}
\begin{cventries}
    \cventry
        { DevOps Engineer III (Engineering --- Lead) }
        { Ezurio, Inc. (Formerly Laird Connectivity, Inc.) }
        { Akron, OH }
        { Jan. 2023 --- Jul. 2024 }
        {
            \begin{cvitems}
                \item{\textbf{Oversaw and collaborated with a transformation partner} to assess DevOps maturity and \textbf{map value streams} across the engineering organization, delivering a roadmap that launched a company-wide \textbf{DevOps Modernization initiative and Agile transformation}.}
                \item{\textbf{Architected and implemented an Internal Development Platform (IDP)} with Spacelift, Ansible, Terraform, Docker, Devcontainers, Packer, and GitHub Actions, incorporating \textbf{shift-left security} features (SCA, CSPM, SAST, DAST, policy-as-code), \textbf{GitOps} release management, and real-time cost monitoring while ensuring \textbf{SLSA level 3} artifact provenance for end-to-end software supply chain integrity.\hfill{\href{https://github.com/jhatler/janus}{\textbf{{\color{awesome}Git}Hub Project:} jhatler/janus}}}
                \item{\textbf{Migrated all on-prem DevOps systems, data, and pipelines to AWS} through leveraging the IDP, retiring risky on-prem infrastructure, \textbf{saving \$21,000/year} in infrastructure costs, \textbf{reducing DevOps workload by 50\%}, and enhancing system resilience; \textbf{effectively reducing high-severity, reactive issue resolution} in favor of a more deliberate, error-budget-aware approach.}
                \item{\textbf{Facilitated agile coaching} across engineering and PMO, fostering an iterative Plan-Do-Check-Act (PDCA) approach, cultivating an Agile Community of Practice, and aligning enterprise teams around new tools and processes to \textbf{ensure scaling of DevOps principles and responsiveness to evolving team needs} throughout the transformation.}
                \item{\textbf{Onboarded and mentored a new DevOps Engineer} to gain deep knowledge of our diverse ecosystem and become the primary DevOps SME for the organization, \textbf{ensuring continuity of expertise within 18 months}.}
            \end{cvitems}
        }
\end{cventries}
\begin{cventries}
    \cventry
        { DevOps Engineer III (IT) }
        { Laird Connectivity, Inc. }
        { Akron, OH }
        { Jul. 2019 --- Jan. 2023 }
        {
            \begin{cvitems}
                \item{\textbf{Engineered a custom Jenkins (Groovy) framework for embedded Linux BSPs} (Buildroot/Yocto) to declaratively manage build parameters and dependencies \textbf{across 1000+ Git repositories}, enabling advanced versioning and branching strategies, unifying developer experience, \textbf{reducing toil, and increasing deployment frequency} from the order of days to hours.}
                \item{\textbf{Orchestrated a comprehensive domain migration} for all DevOps systems, developing a suite of custom tools for data analysis, staging/UAT setup, migration, validation, and rollback to ensure every reference (documentation, 1500+ Git repos, code reviews, issues) was transparently updated and referential integrity was maintained \textbf{without impacting any release cadence during corporate spin-off} (exceeding SLOs).}
                \item{\textbf{Established a 200-core, 1.5TB RAM bare-metal cluster} with Rancher (Kubernetes), Harvester, Datadog, Docker, and iPXE to which all high-impact CI/CD workloads were migrated, \textbf{minimizing performance contention and decreasing lead time for changes} through increased parallelism to align with multi-year strategic goals.}
                \item{\textbf{Organized ``Lunch \& Learn'' sessions} averaging approx. 30 participants per event, intermixing guest speakers with a series of Git Clinics I developed covering deep dives on Git features (rebasing, cherry-picking, etc.) and \textbf{reinforcing DevOps best practices}.}
                \item{\textbf{Organized and directed a 12-hour DevOps Modernization Workshop} that brought together product managers, executives, and technical leads to align on topics synthesized from the DevOps Research and Assessment \textbf{(DORA)} framework, The Open Group Architecture Framework \textbf{(TOGAF)}, The Open Group Service Integration Maturity Model \textbf{(OSIMM)}, the Digital Practitioner Body of Knowledge \textbf{(DPBoK)}, and DevSecOps research from Carnegie Mellon University Software Engineering Institute \textbf{(CMU-SEI)}, successfully \textbf{culminating in executive buy-in} to onboard a specialized transformation partner \textbf{for ambitious development platform upgrades to accelerate time-to-market}.}
            \end{cvitems}
        }
\end{cventries}
\begin{cventries}
    \cventry
        { DevOps Engineer II (IT) }
        { Laird Connectivity, Inc. (Laird PLC Spin-off) }
        { Akron, OH }
        { Apr. 2018 --- Jul. 2019 }
        {
            \begin{cvitems}
                \item{\textbf{Oversaw an in-house migration} of digital marketing systems to AWS, performing a comprehensive architectural/security review of the existing vendor solution, \textbf{orchestrating the migration without downtime}, successfully integrating Dynatrace for observability, \textbf{improving site reliability under high-traffic conditions, and accelerating time-to-market for new campaigns}.}
                \item{\textbf{Developed custom tools to convert 300+ PlasticSCM repositories to Git}, preserving full history/metadata, \textbf{saving \$5k/year} on licensing, and enabling the professional services team to integrate into the central DevOps platform.}
                \item{\textbf{Authored and delivered an on-site Git training curriculum for 30+ engineers}, enhancing collaboration with the wider engineering organization and \textbf{motivating usage of shared automation}.}
                \item{\textbf{Owned the DevOps platform during two business-unit divestitures}, leveraging Infrastructure as Code to create parallel systems \textbf{without additional capex} and implementing Jira Service Desk to support ITSM within the organization, \textbf{exceeding SLOs with zero support tickets} under Transition Service Agreements.}
            \end{cvitems}
        }
\end{cventries}
\begin{cventries}
    \cventry
        { DevOps Engineer I (IT) }
        { Laird PLC }
        { Akron, OH }
        { May. 2016 --- Apr. 2018 }
        {
            \begin{cvitems}
                \item{\textbf{Transformed an undocumented legacy DevOps ecosystem} into a centralized platform (Jenkins, GitLab, Bugzilla, JFrog Artifactory, MediaWiki) by documenting workflows, creating job aids, tightening integrations, and executing a \textbf{zero-downtime P2V migration}.}
                \item{\textbf{Expanded DevOps systems to support 100+ newly acquired staff}, migrating data from Jira, SVN, and Redmine into Bugzilla, GitLab, and other platform tools and \textbf{coaching teams to adopt standardized processes}.}
                \item{\textbf{Engineered containerized DevOps environments} (Docker, Packer, Terraform, Rancher) to \textbf{support M\&A activities and ensure business continuity}, achieving rapid recovery during a hyperconverged infrastructure failure with no data loss or missed deadlines.}
                \item{\textbf{Engineered cached Android BSP pipelines} (Docker, BTRFS, Jenkins) following extensive analysis of the Android build system and load testing of available infrastructure, successfully \textbf{cutting build times by 80\% and reducing storage consumption by 40\%} while preserving artifact integrity.}
            \end{cvitems}
        }
\end{cventries}
\begin{cventries}
    \cventry
        { AWS Systems Administrator }
        { My1HR, Inc. }
        { Akron, OH }
        { Mar. 2013 --- Feb. 2016 }
        {
            \begin{cvitems}
                \item{\textbf{Architected and scaled a secure, multi-tier AWS infrastructure} (EC2, RDS, ELB, S3, CloudFront) with \textbf{layered security controls} (IDP, DLP, SIEM) and \textbf{tightly integrated observability tools} (ELK, MySQL NDB Cluster Manager, NewRelic, Zabbix, Nagios) that powered a high-traffic, multi-tenant, HIPAA/HITECH/PCI-DSS compliant SaaS product which helped over \textbf{1M+ families annually} to enroll in healthcare plans --- \textbf{exceeding SLAs by maintaining five-nines availability during peak seasonal traffic}.}
                \item{\textbf{Diagnosed and resolved critical MySQL NDB Cluster CGE crashes} (SEGFAULTS) by analyzing core dumps to pinpoint a memory page alignment bug and working with Oracle to deploy a patch within one week, \textbf{maintaining all SLOs and SLAs during the process}.}
                \item{\textbf{Streamlined infrastructure management} by migrating from Chef/RightScale to CloudFormation/Puppet, improving autoscaling agility by 40\% and \textbf{saving \$50k/year}.}
                \item{\textbf{Evangelized Docker for local development}, standardizing tooling across a 30-person team and \textbf{saving 20+ hours} during developer onboarding.}
            \end{cvitems}
        }
\end{cventries}




%-------------------------------------------------------------------------------
\end{document}
